\documentclass[a4paper,15pt]{exam}
\usepackage{kotex}
\usepackage{geometry}
\usepackage{amsmath}
\usepackage{tikz}
\usetikzlibrary{positioning}
\usepackage{scalerel,amssymb}
\usepackage{calc}
\usepackage{pst-3dplot}
\usepackage{amsfonts} % for heart symbol
\usepackage{xcolor} % for color box
\geometry{margin=0.5cm}

\def \mbox {\begin{tikzpicture} \draw [draw=black](0,0) rectangle (0.3,0.3);\end{tikzpicture}}

\begin{document}

\fontsize{13pt}{24pt}\selectfont

\begin{center}
  \bfseries\LARGE

  수학 \#2
  \bigskip
  \normalfont\normalsize
\end{center}


\begin{questions}
    \question
        다음 조건에 맞는 수를 모두 찾아 쓰세요. 

        \begin{center}
            \fbox{\fbox{\parbox{7cm}{
                조건 1. 각 숫자의 합이 4입니다.\\
                조건 2. 두 자리 수입니다. 
            }}}
        \end{center}

        \vspace{\stretch{1}}


    \question
        다음 숫자 카드로 만들 수 있는 두 자리 수 중에서 두 번째로 작은 두자리 수를 쓰세요. 
        
        \begin{center}
            \fbox{9} \fbox{0} \fbox{4} \fbox{1}
        \end{center}

        \vspace{\stretch{1}}


    \question
        다음 숫자 카드 중 2장을 골라 두 자리 수를 만들려고 합니다. 두 자리 수중 20보다 크고 75보다 작은 수의 개수를 구하세요
        
        \begin{center}
            \fbox{5} \fbox{0} \fbox{2} \fbox{7}
        \end{center}

        \vspace{\stretch{1}}

    \newpage

    \question
        숫자 \textbf{43}을 아래와 같이 설명했습니다. \mbox{ } 안에 들어갈 말을 쓰세요. 

        \begin{center}
            \fbox{\fbox{\parbox{12cm}{
                \mbox{ } 자리 수 입니다. \\
                \mbox{ } 의 자리 숫자에서 일의 자리 숫자를 빼면 \mbox{ } 입니다. \\
                십의 자리 숫자와 일의 자리 숫자의 합은 \mbox{ } 입니다.
            }}}
        \end{center}

        \vspace{\stretch{1}}

    \question
        다음 숫자 카드 중 2장을 골라 두 자리 수를 만들었습니다. 만든 수 중에서 세 번째로 큰 두 자리 수가 97일 때, 뒤집어진 숫자 카드에 적힌 숫자를 구하세요.
        
        \begin{center}
            \fbox{9} \fbox{7} \colorbox{black}{0} \fbox{8}
        \end{center}

        \vspace{\stretch{1}}

\end{questions}
\end{document}
