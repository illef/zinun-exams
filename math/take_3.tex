\documentclass[a4paper,15pt]{exam}
\usepackage{kotex}
\usepackage{geometry}
\usepackage{amsmath}
\usepackage{tikz}
\usetikzlibrary{positioning}
\usepackage{scalerel,amssymb}
\usepackage{calc}
\usepackage{pst-3dplot}
\usepackage{amsfonts} % for heart symbol
\usepackage{xcolor} % for color box
\usepackage{csquotes} % for quotes
\geometry{margin=0.5cm}

\def \mbox {\begin{tikzpicture} \draw [draw=black](0,0) rectangle (0.3,0.3);\end{tikzpicture}}

\begin{document}

\fontsize{13pt}{24pt}\selectfont

\begin{center}
  \bfseries\LARGE

  수학 \#3
  \bigskip
  \normalfont\normalsize
\end{center}

\noindent\fbox{
    \parbox{\textwidth}{
    문장을 \textbf{식}으로 표현하는 연습을 합니다.

    \begin{enumerate}
        \item \enquote{3과 4를 더하면 9 입니다.} 라는 문장은 $\boldsymbol{3+4=9}$ 로 표현할 수 있습니다.
        \item \enquote{어떤 수에 3을 더하니 10이 되었습니다.} 라는 문장은 $\boldsymbol{x+3=10}$ 로 표현할 수 있습니다.
        \item \enquote{두 수를 더하니 10이 되었습니다.} 라는 문장은 $\boldsymbol{x+y=10}$ 로 표현할 수 있습니다.
        \item \enquote{22를 6번 곱하면 어떤 수가 될까요?} 라는 문장은 $\boldsymbol{22\times 6=y}$ 로 표현할 수 있습니다.
    \end{enumerate}
    }
}



\begin{questions}
    \question
        \enquote{어떤 수에서 커지는 규칙으로 100씩 6번 뛰어 세었더니 5814가 되었습니다} 를 식으로 표현하세요
        \vspace{\stretch{1}}


    \question
        \enquote{어떤 수에서 작아지는 규칙으로 10씩 5번 뛰어 세었더니 2902가 되었습니다.} 를 식으로 표현하세요
        \vspace{\stretch{1}}


    \question
        \enquote{어떤 수에서 커지는 규칙으로 100씩 4번 뛰어 세었더니 8685가 되었습니다. 어떤 수에서 작아지는 규칙으로 10씩 7번 뛰어 센 수는 무엇입니까?} 를 식으로 표현하세요
        \vspace{\stretch{1}}


    \question
        \enquote{매일 400원씩 모아 3200짜리 장난감을 사려고 합니다. 몇일 모아야 할까요?} 를 식으로 표현하세요
        \vspace{\stretch{1}}

    \newpage

    \question
        \enquote{하영이는 1000원짜리 지폐 4장과 100원짜리 동전 30개를 가지고 있습니다. 하영이가 가진 돈은 모두 얼마입니까?} 를 식으로 표현하고 하영이가 가진 돈이 얼마인지 맞춰보세요
        \vspace{\stretch{1}}

    \question
        \enquote{저금통에 1000원짜리 지폐 2장, 500원짜리 동전 5개, 100원짜리 동전 15개가 들어 있습니다. 저금통에 들어 있는 돈은 얼마입니까?} 를 식으로 표현하고 저금통에 들어 있는 돈을 맞춰보세요
        \vspace{\stretch{1}}

    \question
    \enquote{어떤 수에서 커지는 규칙으로 300씩 3번 뛰었더니 7320이 되었습니다. 어떤 수에서 커지는 규칙으로 30씩 3번 뛰어 센 수는 무엇입니까?} 를 식으로 표현하세요
        \vspace{\stretch{1}}

    \question
    어떤 수에서 커지는 규칙으로 40씩 3번 뛰어 세어야 하는데 실수로 30씩 3번 뛰어 세었더니 390 이 되었습니다. 바르게 뛰어 센 수는 얼마입니까?
        \vspace{\stretch{1}}


\end{questions}
\end{document}
