\documentclass[a4paper,15pt]{exam}
\usepackage{kotex}
\usepackage{geometry}
\usepackage{amsmath}
\usepackage{tikz}
\usetikzlibrary{positioning}
\usepackage{scalerel,amssymb}
\usepackage{calc}
\usepackage{pst-3dplot}
\usepackage{amsfonts} % for heart symbol
\usepackage{xcolor} % for color box
\geometry{margin=0.5cm}

\def \mbox {\begin{tikzpicture} \draw [draw=black](0,0) rectangle (0.3,0.3);\end{tikzpicture}}

\begin{document}

\fontsize{13pt}{24pt}\selectfont

\begin{center}
  \bfseries\LARGE

  수학 \#4
  \bigskip
  \normalfont\normalsize
\end{center}


\begin{questions}
\question
    \mbox{ } 안에 들어갈 수와 계산 결과를 쓰시오.
    \begin{parts}
        \part 
            $ 6 \times 3 = \mbox{ } + \mbox{ } + \mbox{ } $
            \answerline
        \part 
            $ 3 \times 6 = ( 3 \times 1 ) + ( 3 \times \mbox{ } ) $
            \answerline
        \part 
            $ 3 \times 6 = 3 + ( 3 \times \mbox{ } ) $
            \answerline
        \part 
            $ 3 \times 6 = 6 + ( 3 \times \mbox{ } ) $
            \answerline
    \end{parts}
    \vspace{\stretch{1}}


\question
    아례 식의 계산 결과를 쓰시오.
    \begin{parts}
        \part 
            $ 27 \times 3 $
            \answerline
        \part 
            $ 47 \times 4 $
            \answerline
        \part 
            $ 120 \times 20 $
            \answerline
        \part 
            $ 120 \times 22 $
            \answerline
        \part 
            $ 157 \times 101 $
            \answerline
    \end{parts}
    \vspace{\stretch{1}}

\newpage

\question
    다음 숫자 카드로 만들 수 있는 두 자리 수 중에서 세 번째로 작은 두자리 수를 쓰시오.
    
    \begin{center}
        \fbox{9} \fbox{0} \fbox{4} \fbox{1}
    \end{center}

    \vspace{\stretch{1}}


\question
    다음 숫자 카드 중 2장을 골라 두 자리 수를 만들었습니다. 만든 수 중에서 세 번째로 작은 두 자리 수가 13 일 때, 뒤집어진 숫자 카드에 적힌 숫자를 구하세요.
    
    \begin{center}
        \fbox{1} \fbox{3} \colorbox{black}{0} \fbox{2}
    \end{center}

    \vspace{\stretch{1}}


\question
    \enquote{어떤 수에서 커지는 규칙으로 300씩 3번 뛰었더니 6900이 되었습니다. 어떤 수에서 커지는 규칙으로 30씩 3번 뛰어 센 수는 무엇입니까?} 를 식으로 표현하세요. 문자로 x와 y를 사용하세요.
        \vspace{\stretch{1}}

\question
    어떤 수에서 커지는 규칙으로 40씩 3번 뛰어 세어야 하는데 실수로 20씩 3번 뛰어 세었더니 260 이 되었습니다. 바르게 뛰어 센 수는 얼마입니까?
        \vspace{\stretch{1}}


\end{questions}
\end{document}
